%%% Glossar erstellen (Package Glossaries)
%
% Syntaxinfo - Glossareintrag:
% \newglossaryentry{<label>}{<settings>}
% 
% <label> setzt die spezifische Abkürzung nach der im Text referenziert werden kann (z.B. mit \gls{<label>})
%
% mögliche <settings>:
% name = {<text>}    setzt den Namen der im Text als Abkürzung angezeigt wird
% description = {<text>}    Ist die Erklärung der Abkürzung
% sort = {<text>}    wenn bei der Sortierung des Glossars dieser Text statt der Abkürzung verwendet werden soll
% plural = {<text>}    Setzt die Abkürzung mit einem spezifischen Plural
% symbol = {<text>}    Definiert ein bestimmtes Symbol für diese Abkürzung
%
% ------------------------------
% Syntaxinfo - Akronym:
% \newacronym{<label>}{<abbrv>}{<full>}
%
% <label> setzt die spezifische Abkürzung nach der im Text referenziert werden kann (z.B. mit \gls{<label>})
% <abbrv> setzt den Namen der im Text als Abkürzung angezeigt wird
% <full> Volle Ausschreibung der Abkürzung
%
% Referenzierung
% \gls{<label>}        Standardreferenzierung fuer Glossar und Akronyme
% \glspl{<label>}     Pluralreferenzierung
% \Gls{<label>}       Referenzierung mit großem Anfangsbuchstaben
% \Glspl{<label>}    Pluralreferenzierung mit großem Anfangsbuchstaben
% \acrlong{<label>} Akronym Referenzierung kurz
% \acrshort{<label>}Akronym Referenzierung ausgeschrieben
% \acrfull{<label>}    Akronym volle ausschreibung mit abkuerzung in klammern

\newacronym{plc}{PLC}{Programmable Logic Controller}

\newglossaryentry{bodas}
{
    name={BODAS}, 
    description={\underline{Bo}sch Rexroth \underline{D}esign \& \underline{A}pplication \underline{S}ystem for Mobile Electronics}
}

\newacronym{rc}{RC}{Rexroth Controller}

\newacronym{iec}{IEC}{International Eletrotechnical Commission}

\newacronym{api}{API}{Application Programming Interface}

\newglossaryentry{capi}
{
    name={C-API}, 
    description={An API using C as programming language}
}
\newacronym{ide}{IDE}{Integrated Development Environment}

\newacronym{ecu}{ECU}{Engine Control Unit}

\newglossaryentry{i4}
{
    name={Industry 4.0},
    description={is a project of the german federal government to increase the usage of connected machines and processes (Internet of Things) to develop a \textit{Smart Factory} with intelligent monitoring and autonomous decision-making processes. This is supposed to be the fourth industrial revolution after Industry 1.0 (steam-engine), Industry 2.0 (assembly lines) and Industry 3.0 (\glspl{plc})} \cite{Indu4}
}

\newacronym{cpf}{cpf}{clicks per function}

\newacronym{gui}{GUI}{Graphical User Interface}

\newglossaryentry{os}
{
    name={Operating System},
    description={is software that manages the resources of a computer and provides them to other software that is running on the system}
}

\newacronym{cpu}{CPU}{Central processing unit}

\newacronym{ram}{RAM}{Random-access memory}

\newglossaryentry{watchdog}
{
    name={Watchdog},
    description={or watchdog unit is a hardware based system to monitor internal signals and performances, mostly used for safety functions, to prevent the system of getting in a critical status. Therefor it can fire up predefined tasks that can safely bring the system in a secure status.}
}

\newglossaryentry{can}
{
    name={CAN},
    description={stands for controller area network and is a vehicle bus standard originally developed by Robert Bosch GmbH. It enables a communication within the vehicle between controllers and other electrical components by directly connecting them.}
}

\newglossaryentry{uart}
{
    name={UART},
    description={(universal asynchronous receiver/transmitter) is an hardware internal component that is used to translate between serial and parallel data and therefor often seen as an interface between external communication and internal processing.}
}

\newacronym{iop}{I/O Port}{Input/Output Port}

\newglossaryentry{eep}
{
    name={EEPROM},
    description={stands for Electrically Erasable Programmable Read-Only Memory and is a non-volatile memory unit to store parameter and other data which is required to stay in the memory regardless of the system's power supply}
}

\newacronym{html}{HTML}{Hypertext Markup Language}

\newglossaryentry{eou}
{
    name={BODAS-design Ease of Use},
    description={is a way of using the CoDeSys based BODAS-design to develop user applications for Rexroth controllers using predefined blocks that already contain logic and hardware based code snippets. This system is supposed to make developing easier to understand by combining blocks and setting parameter of the application without writing the actual code.}
}

\newglossaryentry{eouzone}
{
    name={BODAS-design Ease of Use Zone model},
    description={offers a file that can be loaded in the controller (.hex file) that divide the in- and outputs of the controller in zones that can, depending on what pins are assigned to the zone, provide specific functionality by selecting from precoded user applications. The configuration and parametrization is done within the Rexroth service and diagnose tool BODAS service.}
}

\newglossaryentry{hex-file}
{
    name={.hex file},
    description={stores machine language code in a hexadecimal form. It can be directly loaded (flashed) into the controller.\cite{Daga}},
    sort=hex
}

\newglossaryentry{asrun}
{
    name={ASrun Application},
    description={is a, mostly in a .hex file offered, user application that is ready to use by configuring it via the Rexroth service tool BODAS service without programming.}
}

\newacronym{rc45}{RC4-5/30}{Rexroth Controller Type 4-5 Series 30}

\newacronym{bu}{Bu}{BODAS utilities}

\newacronym{bs}{Bs}{BODAS service}

\newglossaryentry{wysiwyg}
{
    name={WYSIWYG-Editor},
    description={is a text editing program in which the written text is directly shown in the way it is supposed to look in a printed or browser computed version.}
}

\newacronym{msword}{MS Word}{Microsoft Word}

\newacronym{id}{ID}{Identifier}

\newacronym{dsd}{DSD}{Dual Solenoid Driver}

\newacronym{afc}{AFC}{Automatic Fan Control}

